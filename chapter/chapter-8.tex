\chapter{結論 \label{c8}}

\section{まとめ}
本研究チームでは,話しことばか書きことばの判断が分かれるグレーゾーン表現に対し,話しことばチェッカーがより柔軟に対応できるようにするため,機械学習と組み合わせたハイブリッドなシステムの構築を目標にあげている.先行研究では,グレーゾーンの1つである「てしまう」に特化した分類モデルを構築したが,機械学習モデル構築時には,大量の学習データや検証データが要求されるうえに,専門家の知識が必要となるラベル付けの工程もあり,「てしまう」以外のグレーゾーンに対応できるようにするには,膨大な労力が必要となる.本研究では,機械学習モデル構築までの工程の整理や,ツールの構築を通した事例収集の効率化を検討し,近年台頭しているLLMを活用したラベル付けの方策を模索した.

検証では,LLMによるグレーゾーンの主観客観分類を行い,先行研究で構築された分類モデルと同等の精度で分類できることがわかり,LLMによるグレーゾーン分類は実用可能性があることが考えられる.

LLMによる話しことば検出では,話しことばの例や,書きことばリストをプロンプトに与えたり,話しことば・書きことばそれぞれの定義を与えることで,LLMでも話しことば検出は可能であることがわかったが,それでも話しことばチェッカーが話しことばと認識しない出力結果もあった.これは,プロンプトに与えた話しことば・書きことばの定義が関係していると考えられ,本研究チームの定義をより具体的なものにして使用すれば,話しことばチェッカーに近い検出が可能になると考えられる.また,専門家の判定では,LLMの検出結果はほとんどグレーゾーンであり,新たな話しことば検出には至れなかったが,前述の通りプロンプトの整備を重ねることで,話しことば検出が可能になると考えられる.

\section{今後の展望}
今後の展望として,LLMによるグレーゾーンの事例文生成が考えられる.本研究では,データ収集の効率化の検討や,LLMによる話しことば検出,先行研究の仮定に基づくグレーゾーンの主観客観判定などの性能調査でとどまっている.今後は,グレーゾーンを含む主観的または客観的な文章の生成を行えるようにプロンプトを整備し,データ収集の効率化が実現できると考えられる.

また,プロンプトに与える話しことばや書きことばの情報をより詳細なものに改良していくことで,話しことばチェッカーと同様の検出が可能になり,入力文を利用した修正が可能になると考えられる.システムでは登録されている修正例を提示していたが,学生が執筆した文章をもとに修正例の提示を行う機能の実装が考えられる.

また,LLMが提示した表現が専門家によって新たな話しことばと認められた場合は,システムの仕様上,話しことばデータベースに登録する必要がある.今後も話しことばとされる表現が変化していくことを想定したシステム運用も検討することが重要であると考えられる.

% しかし,LLMの生成物には事実とは異なる記述が含まれるハルシネーションと呼ばれる現象があり,