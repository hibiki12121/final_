\chapter{妥当性への脅威 \label{c8}}

\section{外的妥当性への脅威 \label{c8s1}}
本研究で使用したLLMについて,第\ref{c6}章ではブラウザでチャット形式で入力を行うChatGPT-3.5,第\ref{c7}章で使用したLLMはGPT-4のAPIを使用しており,それぞれ別のバージョンでの検証は行っていない.これは外的妥当性への脅威になると考えられる.それぞれの検証について互いに使用していないバージョンのGPTを使用することで,GPTの各検証ごとの有用性が評価できると考えられる.

第\ref{c7}章で使用したデータは,2019年度の1つの講義内でのレポートを使用し,さらにその講義で提出されたレポート全体から16件をデータとして使用した.これはデータが持つ範囲として極めて限定されているため,外的妥当性への脅威になると考えられる.レポートで使われる表現は学生によって異なるため,データとして使用するレポートを変更することで得られる結果が変化することは容易に考えられる.外的妥当性を高めるために,提出されたレポート全てをデータとして使用すること,他講義のレポートをデータに含めることが挙げられる.

\section{内的妥当性への脅威}
第\ref{c6}章では,「てしまう」を含む文章の主観客観分類を,第\ref{c7}章では,レポート文に含まれる話しことば検出をそれぞれLLMに行わせている.

第\ref{c6}章で行った「てしまう」を含む文章の主観客観分類で使用した文章は,ラベルが主観であるものを40(20)件ずつ,客観であるものを40(20)件ずつ分ける形式でプロンプトに入力していた.これにより,LLMが分類時に主観文および客観文の特徴を掴み,精度に影響を及ぼしている可能性がある.この観点で内的妥当性への脅威が存在すると考えられる.本研究では主観または客観とラベル付けされた文章を混合させた場合の検証は行っていない.内的妥当性を高めるためには,前述の状況での検証を行うことが必要であると考えられる.

第\ref{c7}章で行ったレポート文章から話しことばを検出させる検証において,プロンプトに入力していた話しことばの定義は,山下や過去の研究で述べられていたものを筆者が編集したものである.これは筆者の編集次第で得られる結果が変化する可能性があるため,内的妥当性への脅威になると考えられる.また,書きことばの定義についても同様に,定義の仕方で得られる結果が変わる可能性がある.特に,話しことばは\ref{c3s3}節で述べたように,関連書籍や実際のレポートを用いて事例集という形式でまとめられているため,外延的な定義としてみなすことができる.また,過去の話しことばチェッカーに関する研究で述べられている話しことばの定義についての文言がわずかに異なっており,内包的な定義として共通したものが存在しているのか不明瞭である.今後,話しことばに関する検証でLLMを用いる場合は,内包的な定義を明確にする必要があると考えられる.