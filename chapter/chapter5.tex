\chapter{LLMによる主観客観判定 \label{c5}}

\section{5.2 \label{c5s2}}
LLMは説明不可能なAIであるため,話しことば/書きことばの分類が有効であるかの妥当性を判断することは難しい.そこで,先行研究で使用したBERTモデルによるグレーゾーンを含む文章の主観客観分の分類タスクをLLMに行わせた.

本検証では,BERTモデル構築の時に使われた山下が作成したグレーゾーン「てしまう」を含む例文を集めたデータセットを用いた.これに対し,以下のようなプロンプトを与え,LLMによるグレーゾーンの分類を行わせた.

\begin{table}[H]
\centering
\caption{データセットより抜粋}
\begin{tabular}{|c|l|}
\hline
\multicolumn{1}{|c|}{主観/客観} & \multicolumn{1}{c|}{例文} \\ \hline
主観 & \begin{tabular}[l]{@{}l@{}}以前はゴルフのプレー中にコンタクトが取れてしまうことがあり、\\いつも目のことを気にしていなければならなかった。\end{tabular} \\ \hline
主観 & \begin{tabular}[l]{@{}l@{}}無神経な発言にいちいち腹が立ってしまう自分も大人気ないと思う。\end{tabular} \\ \hline
主観 & \begin{tabular}[l]{@{}l@{}}あせって解約して、思わぬ損をしてしまった。\end{tabular} \\ \hline
主観 & \begin{tabular}[l]{@{}l@{}}言うつもりではなかったのに口をすべらせてしまうことがある。\end{tabular} \\ \hline
客観 & \begin{tabular}[l]{@{}l@{}}絵文字・顔文字は機種間によって表示方法に差異が生じてしまうため、\\自分の感情が正しく受け取る相手に伝えられるとは限らない。\end{tabular} \\ \hline
客観 & \begin{tabular}[l]{@{}l@{}}急いで返信する必要はないと感じると面倒と感じ、特に返信が\\「不必要」と感じたり「催促されている」と感じる際にLINEを使用する\\ことで精神的に疲れを感じてしまうのではないかと考えられる。\end{tabular} \\ \hline
客観 & \begin{tabular}[l]{@{}l@{}}すぐに返信が来ると焦ってしまう、返信する立場のことを考えていない\\ように感じ、余計に返信をする気をなくしてしまうという回答もあった。\end{tabular} \\ \hline
客観 & \begin{tabular}[l]{@{}l@{}}このことにより、返信の催促をされ、受け取った側も精神的に疲れを\\感じLINE使用自体が面倒と感じてしまうようになると考えられる。\end{tabular} \\ \hline
\end{tabular}
\label{ambiguous-sbj-obj}
\end{table}

\subsubsection{検証方法}
検証は,以下の項目で行った.
\begin{itemize}
    \item[1.] 主観または客観に分類されている文章を20件ずつ,LLMに主観または客観に分類させる
    \item[2.] 主観または客観に分類されている文章を40件ずつ,LLMに主観または客観に分類させる
    \item[3.] 主観または客観に分類されている文章を20件ずつ,LLMに話しことばまたは書きことばに分類させる
    \item[4.] 主観または客観に分類されている文章を40件ずつ,LLMに話しことばまたは書きことばに分類させる
\end{itemize}
話しことばまたは書きことばへの分類は,先行研究\cite{checker}の仮定を採用している.

ChatGPTのブラウザ版を使用し,表\ref{prompt-spokenorwritten}と山下が作成したデータセット475件を40件ずつに分けて回答を生成させたとき,20件ずつに分けて回答を生成させたときのそれぞれで行った.
プロンプトには,ChatGPTが出力を生成する際に条件を認識しやすくするために役割を与えている.これは,ChatGPTが広範囲の分野を学習しており,回答を生成するときの一貫性や回答精度を向上させる目的で与えている.

\begin{table}[H]
\centering
\caption{主観客観分類のためのプロンプト(例の組数=3のとき)}
\small % \footnotesize
\begin{tabular}{|l|}
\hline
\multicolumn{1}{|c|}{プロンプト} \\ \hline
\begin{tabular}[c]{@{}l@{}} 
        \#\#\# 役割 \#\#\#\\
        あなた(GPT)を,「大学初年次における日本語文章教育のエキスパート」とします.\\
        \\
        \#\#\# 指示 \#\#\#\\
        「判定する文章」の中の文章それぞれを,主観的であるか客観的であるかを判定してくだ\\さい.\\文章は1文ずつ「。」で区切って提示するので,それぞれについて判定をしてください.\\出力は「フォーマット」に従ってください.「例」では判定理由は省略しています.\\
        <判定理由>は1~2文程度で記述してください.\\
        \\
        \#\#\# 例 \#\#\#\\
        レシートをポケットに入れたまま選択してしまい、洗濯機の掃除に苦労したことがある。\\
        主観的\\
        \\
        風邪を引いてしまい、追試を受けなければならなくなった。\\
        主観的\\
        \\
        病院で処方された薬を飲み忘れてしまったことがある。\\
        主観的\\
        \\
        提出日時に一秒でも遅れてしまうと提出できないようシステム設計されている。\\
        客観的\\
        \\
        Zoomの場合受講生数が多くなると、一画面に収まらなくなり、複数の画面へと分割\\されてしまう。\\
        客観的\\
        \\
        様々な事情により孤独に陥ってしまった人への自治体の支援が十分行き届いていない\\のが現状である。\\
        客観的\\
        \\
        \#\#\# フォーマット \#\#\#\\
        <判定した文章>\\
        <判定結果>\\
        % <判定理由>\\
        \\
        \#\#\# 判定する文章 \#\#\#\\
        「いいね」の数にこだわることは無意味だとはわかっていても、やはり「いいね」の数は\\気になってしまう。\\ \\
        SNSに来たメッセージには「早く返さなければ」と思うため、メッセージを確認する\\ために常にスマホを見てしまうという傾向がある。\\ \\
        スマホ利用実態調査の実験中、「スマホを操作してはならない」という抑圧される\\苦しさに何度も陥ってしまうことがあったが、それでは脱依存を試みることは難しいこと\\も体験した。\\

\end{tabular}   \\ \hline

\end{tabular}
\label{prompt-spokenorwritten}
\end{table}



以上の検証方法を,Zero-shot Prompting, One-shot Prompting, Few-shot Prompting(n=13) の手法で行い,例の件数を変えたことによるChatGPTの回答結果を比較した。

\subsection{Zero-shot Promptingでの分類の結果}
Zero-shot Promptingでの回答結果の評価を表\ref{cfm-ex0}に示す。各々の条件での混合行列は表\ref{cf-ex0-sw20}、表\ref{cf-ex0-sw40}、表\ref{cf-ex0-so20}、表\ref{cf-ex0-so40}に示す。
\begin{table}[H]
\centering
% \small
\caption{Zero-shot Promptingでの回答結果の評価}
\begin{tabular}{|l|r|r|r|r|}
\hline
\multicolumn{1}{|c|}{} & \multicolumn{1}{c|}{\begin{tabular}[c]{@{}c@{}}話しことば\\書きことば\\ 分類(20件ずつ)\end{tabular}} & \multicolumn{1}{c|}{\begin{tabular}[c]{@{}c@{}}話しことば\\書きことば\\ 分類(40件ずつ)\end{tabular}} & \multicolumn{1}{c|}{\begin{tabular}[c]{@{}c@{}}主観/客観\\ 分類(20件ずつ)\end{tabular}} & \multicolumn{1}{c|}{\begin{tabular}[c]{@{}c@{}}主観/客観\\ 分類(40件ずつ)\end{tabular}} \\ \hline
Accuracy    & 64.26\% & 75.32\% & 64.68\% & 66.26\% \\ \hline
Precision   & 64.63\% & 85.63\% & 65.61\% & 66.26\% \\ \hline
Recall      & 62.98\% & 60.85\% & 61.70\% & 66.26\% \\ \hline
F1          & 63.79\% & 71.14\% & 63.60\% & 66.26\% \\ \hline
\end{tabular}
\label{cfm-ex0}
\end{table}
\begin{table}[H]
\centering
\caption{話しことば/書きことば分類(20件ずつ)の混同行列}
\begin{tabular}{|c|r|r|}
\hline
 & \multicolumn{1}{c|}{\begin{tabular}[c]{@{}c@{}}ChatGPT\\ 書きことば\end{tabular}} & \multicolumn{1}{c|}{\begin{tabular}[c]{@{}c@{}}ChatGPT\\ 話しことば\end{tabular}} \\ \hline
\begin{tabular}[c]{@{}c@{}}山下データセット\\ 書きことば\end{tabular} & 148 & 87 \\ \hline
\begin{tabular}[c]{@{}c@{}}山下データセット\\ 話しことば\end{tabular} & 81 & 154 \\ \hline
\end{tabular}
\label{cf-ex0-sw20}
\end{table}

% 使用箇所: c6s1-1
\begin{table}[H]
\centering
\caption{話しことば/書きことば分類(40件ずつ)の混同行列}
\begin{tabular}{|c|r|r|}
\hline
 & \multicolumn{1}{c|}{\begin{tabular}[c]{@{}c@{}}ChatGPT\\ 書きことば\end{tabular}} & \multicolumn{1}{c|}{\begin{tabular}[c]{@{}c@{}}ChatGPT\\ 話しことば\end{tabular}} \\ \hline
\begin{tabular}[c]{@{}c@{}}山下データセット\\ 書きことば\end{tabular} & 143 & 92 \\ \hline
\begin{tabular}[c]{@{}c@{}}山下データセット\\ 話しことば\end{tabular} & 24 & 211 \\ \hline
\end{tabular}
\label{cf-ex0-sw40}
\end{table}
\begin{table}[H]
\centering
\caption{主観/客観分類(20件ずつ)の混同行列}
\begin{tabular}{|c|r|r|}
\hline
 & \multicolumn{1}{c|}{\begin{tabular}[c]{@{}c@{}}ChatGPT\\ 客観\end{tabular}} & \multicolumn{1}{c|}{\begin{tabular}[c]{@{}c@{}}ChatGPT\\ 主観\end{tabular}} \\ \hline
\begin{tabular}[c]{@{}c@{}}山下データセット\\ 客観\end{tabular} & 145 & 90 \\ \hline
\begin{tabular}[c]{@{}c@{}}山下データセット\\ 主観\end{tabular} & 76 & 159 \\ \hline
\end{tabular}
\label{cf-ex0-so20}
\end{table}
\begin{table}[H]
\centering
\caption{主観/客観分類(40件ずつ)の混同行列}
\begin{tabular}{|c|r|r|}
\hline
 & \multicolumn{1}{c|}{\begin{tabular}[c]{@{}c@{}}ChatGPT\\ 客観\end{tabular}} & \multicolumn{1}{c|}{\begin{tabular}[c]{@{}c@{}}ChatGPT\\ 主観\end{tabular}} \\ \hline
\begin{tabular}[c]{@{}c@{}}山下データセット\\ 客観\end{tabular} & 151 & 84 \\ \hline
\begin{tabular}[c]{@{}c@{}}山下データセット\\ 主観\end{tabular} & 84 & 151 \\ \hline
\end{tabular}
\label{cf-ex0-so40}
\end{table}

話しことば/書きことば分類(40件ずつ)のときのAccuracyが75.32\%で最も高く、適合率も85.63\%と、こちらも最も高い結果が得られた。また、混同行列からは、どの条件においても話しことば(主観)であるものを正しく判定できていることがわかる。

\subsection{考察}

\subsection{One-shot Promptingでの分類の結果}
One-shot Promptingでの回答結果の評価を表\ref{cfm-ex1}に示す。各々の条件での混合行列は表\ref{cf-ex1-sw20}、表\ref{cf-ex1-sw40}、表\ref{cf-ex1-so20}、表\ref{cf-ex1-so40}に示す。

\begin{table}[H]
\centering
% \small
\caption{Zero-shot Promptingでの回答結果の評価}
\begin{tabular}{|l|r|r|r|r|}
\hline
\multicolumn{1}{|c|}{} & \multicolumn{1}{c|}{\begin{tabular}[c]{@{}c@{}}話しことば\\書きことば\\ 分類(20件ずつ)\end{tabular}} & \multicolumn{1}{c|}{\begin{tabular}[c]{@{}c@{}}話しことば\\書きことば\\ 分類(40件ずつ)\end{tabular}} & \multicolumn{1}{c|}{\begin{tabular}[c]{@{}c@{}}主観/客観\\ 分類(20件ずつ)\end{tabular}} & \multicolumn{1}{c|}{\begin{tabular}[c]{@{}c@{}}主観/客観\\ 分類(40件ずつ)\end{tabular}} \\ \hline
Accuracy    & 51.91\% & 67.87\% & 62.34\% & 71.91\% \\ \hline
Precision   & 55.42\% & 75.30\% & 62.83\% & 72.29\% \\ \hline
Recall      & 19.57\% & 53.19\% & 60.43\% & 71.06\% \\ \hline
F1          & 28.93\% & 62.34\% & 61.61\% & 71.67\% \\ \hline
\end{tabular}
\label{cfm-ex1}
\end{table}
\begin{table}[H]
\centering
\caption{話しことば/書きことば分類(20件ずつ)の混同行列}
\begin{tabular}{|c|r|r|}
\hline
 & \multicolumn{1}{c|}{\begin{tabular}[c]{@{}c@{}}ChatGPT\\ 書きことば\end{tabular}} & \multicolumn{1}{c|}{\begin{tabular}[c]{@{}c@{}}ChatGPT\\ 話しことば\end{tabular}} \\ \hline
\begin{tabular}[c]{@{}c@{}}山下データセット\\ 書きことば\end{tabular} & 46 & 189 \\ \hline
\begin{tabular}[c]{@{}c@{}}山下データセット\\ 話しことば\end{tabular} & 37 & 198 \\ \hline
\end{tabular}
\label{cf-ex1-sw20}
\end{table}
\begin{table}[H]
\centering
\caption{話しことば/書きことば分類(40件ずつ)の混同行列}
\begin{tabular}{|c|r|r|}
\hline
 & \multicolumn{1}{c|}{\begin{tabular}[c]{@{}c@{}}ChatGPT\\ 書きことば\end{tabular}} & \multicolumn{1}{c|}{\begin{tabular}[c]{@{}c@{}}ChatGPT\\ 話しことば\end{tabular}} \\ \hline
\begin{tabular}[c]{@{}c@{}}山下データセット\\ 書きことば\end{tabular} & 125 & 110 \\ \hline
\begin{tabular}[c]{@{}c@{}}山下データセット\\ 話しことば\end{tabular} & 41 & 194 \\ \hline
\end{tabular}
\label{cf-ex1-sw40}
\end{table}
\begin{table}[H]
\centering
\caption{主観/客観分類(20件ずつ)の混同行列}
\begin{tabular}{|c|r|r|}
\hline
 & \multicolumn{1}{c|}{\begin{tabular}[c]{@{}c@{}}ChatGPT\\ 客観\end{tabular}} & \multicolumn{1}{c|}{\begin{tabular}[c]{@{}c@{}}ChatGPT\\ 主観\end{tabular}} \\ \hline
\begin{tabular}[c]{@{}c@{}}山下データセット\\ 客観\end{tabular} & 142 & 93 \\ \hline
\begin{tabular}[c]{@{}c@{}}山下データセット\\ 主観\end{tabular} & 84 & 151 \\ \hline
\end{tabular}
\label{cf-ex1-so20}
\end{table}
\begin{table}[H]
\centering
\caption{主観/客観分類(40件ずつ)の混同行列}
\begin{tabular}{|c|r|r|}
\hline
 & \multicolumn{1}{c|}{\begin{tabular}[c]{@{}c@{}}ChatGPT\\ 客観\end{tabular}} & \multicolumn{1}{c|}{\begin{tabular}[c]{@{}c@{}}ChatGPT\\ 主観\end{tabular}} \\ \hline
\begin{tabular}[c]{@{}c@{}}山下データセット\\ 客観\end{tabular} & 167 & 68 \\ \hline
\begin{tabular}[c]{@{}c@{}}山下データセット\\ 主観\end{tabular} & 64 & 171 \\ \hline
\end{tabular}
\label{cf-ex1-so40}
\end{table}

\subsection{Few-shot Promptingでの分類の結果}
Few-shot Promptingでの回答結果の評価を表\ref{cfm-ex7}に示す。各々の条件での混合行列は表\ref{cf-ex7-sw20}、表\ref{cf-ex7-sw40}、表\ref{cf-ex7-so20}、表\ref{cf-ex7-so40}に示す。

例の件数は7件と13件のときで検証を行った。

\begin{table}[H]
\centering
% \small
\caption{Few-shot Promptingでの回答結果の評価}
\begin{tabular}{|l|r|r|r|r|}
\hline
\multicolumn{1}{|c|}{} & \multicolumn{1}{c|}{\begin{tabular}[c]{@{}c@{}}話しことば\\書きことば\\ 分類(20件ずつ)\end{tabular}} & \multicolumn{1}{c|}{\begin{tabular}[c]{@{}c@{}}話しことば\\書きことば\\ 分類(40件ずつ)\end{tabular}} & \multicolumn{1}{c|}{\begin{tabular}[c]{@{}c@{}}主観/客観\\ 分類(20件ずつ)\end{tabular}} & \multicolumn{1}{c|}{\begin{tabular}[c]{@{}c@{}}主観/客観\\ 分類(40件ずつ)\end{tabular}} \\ \hline
Accuracy    & 63.40\% & 67.87\% & 64.04\% & 55.74\% \\ \hline
Precision   & 68.00\% & 75.30\% & 70.63\% & 61.34\% \\ \hline
Recall      & 48.51\% & 53.19\% & 48.09\% & 31.06\% \\ \hline
F1          & 57.22\% & 62.34\% & 57.22\% & 41.24\% \\ \hline
\end{tabular}
\label{cfm-ex7}
\end{table}

% 使用箇所: c6s1-2
\begin{table}[H]
\centering
\caption{話しことば/書きことば分類(20件ずつ)の混同行列}
\begin{tabular}{|c|r|r|}
\hline
 & \multicolumn{1}{c|}{\begin{tabular}[c]{@{}c@{}}ChatGPT\\ 書きことば\end{tabular}} & \multicolumn{1}{c|}{\begin{tabular}[c]{@{}c@{}}ChatGPT\\ 話しことば\end{tabular}} \\ \hline
\begin{tabular}[c]{@{}c@{}}山下データセット\\ 書きことば\end{tabular} & 119 & 116 \\ \hline
\begin{tabular}[c]{@{}c@{}}山下データセット\\ 話しことば\end{tabular} & 56 & 179 \\ \hline
\end{tabular}
\label{cf-ex7-sw20}
\end{table}

% 使用箇所: c6s1-2
\begin{table}[H]
\centering
\caption{話しことば/書きことば分類(20件ずつ)の混同行列}
\begin{tabular}{|c|r|r|}
\hline
 & \multicolumn{1}{c|}{\begin{tabular}[c]{@{}c@{}}ChatGPT\\ 書きことば\end{tabular}} & \multicolumn{1}{c|}{\begin{tabular}[c]{@{}c@{}}ChatGPT\\ 話しことば\end{tabular}} \\ \hline
\begin{tabular}[c]{@{}c@{}}山下データセット\\ 書きことば\end{tabular} & 141 & 94 \\ \hline
\begin{tabular}[c]{@{}c@{}}山下データセット\\ 話しことば\end{tabular} & 19 & 216 \\ \hline
\end{tabular}
\label{cf-ex7-sw40}
\end{table}
\begin{table}[H]
\centering
\caption{主観/客観分類(20件ずつ)の混同行列}
\begin{tabular}{|c|r|r|}
\hline
 & \multicolumn{1}{c|}{\begin{tabular}[c]{@{}c@{}}ChatGPT\\ 客観\end{tabular}} & \multicolumn{1}{c|}{\begin{tabular}[c]{@{}c@{}}ChatGPT\\ 主観\end{tabular}} \\ \hline
\begin{tabular}[c]{@{}c@{}}山下データセット\\ 客観\end{tabular} & 113 & 122 \\ \hline
\begin{tabular}[c]{@{}c@{}}山下データセット\\ 主観\end{tabular} & 47 & 188 \\ \hline
\end{tabular}
\label{cf-ex7-so20}
\end{table}
\begin{table}[H]
\centering
\caption{主観/客観分類(20件ずつ)の混同行列}
\begin{tabular}{|c|r|r|}
\hline
 & \multicolumn{1}{c|}{\begin{tabular}[c]{@{}c@{}}ChatGPT\\ 客観\end{tabular}} & \multicolumn{1}{c|}{\begin{tabular}[c]{@{}c@{}}ChatGPT\\ 主観\end{tabular}} \\ \hline
\begin{tabular}[c]{@{}c@{}}山下データセット\\ 客観\end{tabular} & 73 & 162 \\ \hline
\begin{tabular}[c]{@{}c@{}}山下データセット\\ 主観\end{tabular} & 46 & 189 \\ \hline
\end{tabular}
\label{cf-ex7-so40}
\end{table}

% 使用箇所: c6s1-2