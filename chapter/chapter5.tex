\chapter{機械学習モデリングの課題 \label{c5}}

\section{5.1 \label{c5s1}}
機械学習モデルを構築するためには,学習に必要なデータの確保や,データにラベルを付与するラベリングの工程が必要となる.しかし,グレーゾーンを含む文章だけを取り出すことやラベリングにおいては,人の手を介して行わなければならない.前述の通り,グレーゾーンは18種類存在し,各々のグレーゾーンを含む文章のみを用意することは現実的とは言い難い.

グレーゾーンの一つである「てしまう」を含む文のデータセットの構築について,専門家への聞き取り調査を行った.質問は以下の2点である.

\begin{enumerate}
    \item 作成期間
    \item 作成方法
\end{enumerate}

作成期間について,専門家からは,「具体的な期間までは記録しておらず,正確な日数までは計算できないが,概ね1か月程度の期間で作成した」と回答していただいた.作成方法については,「基本的には自身が例文を作成しており,インターネット検索を利用し作成した例もある.また,学生レポートから引用している例もある」と回答を頂いたが,このためだけに「てしまう」を含む文を取り出しておらず,専門家が偶然発見したときに引用すると述べていた.

本研究では,上記の聴き取り調査をもとに,学生が執筆したレポートをデータの確保元としてグレーゾーンを含む例文収集の効率化を図った.

