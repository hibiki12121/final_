\chapter{第5章予定地[検証?] \label{c5}}

\section{5.1 \label{c5s1}}
話しことばチェッカーの現状の課題
グレーゾーンの検出ができていない
機械学習は有効であることが示唆されている。
課題点は事例収集

現在運用している話しことばチェッカーの



\section{5.2 \label{c5s2}}
LLMは説明不可能なAIであるため、話しことば/書きことばの分類が有効であるかの妥当性を判断することは難しい。そこで、先行研究で使用したBERTモデルによるグレーゾーンを含む文章の主観客観分の分類タスクをLLMに行わせた。

本検証では、BERTモデル構築の時に使われた山下が作成したグレーゾーン「てしまう」を含む例文を集めたデータセットを用いた。これに対し、以下のようなプロンプトを与え、LLMによるグレーゾーンの分類を行わせた。

\begin{table}[H]
\centering
\caption{データセットより抜粋}
\begin{tabular}{|c|l|}
\hline
\multicolumn{1}{|c|}{主観/客観} & \multicolumn{1}{c|}{例文} \\ \hline
主観 & \begin{tabular}[l]{@{}l@{}}以前はゴルフのプレー中にコンタクトが取れてしまうことがあり、\\いつも目のことを気にしていなければならなかった。\end{tabular} \\ \hline
主観 & \begin{tabular}[l]{@{}l@{}}無神経な発言にいちいち腹が立ってしまう自分も大人気ないと思う。\end{tabular} \\ \hline
主観 & \begin{tabular}[l]{@{}l@{}}あせって解約して、思わぬ損をしてしまった。\end{tabular} \\ \hline
主観 & \begin{tabular}[l]{@{}l@{}}言うつもりではなかったのに口をすべらせてしまうことがある。\end{tabular} \\ \hline
客観 & \begin{tabular}[l]{@{}l@{}}絵文字・顔文字は機種間によって表示方法に差異が生じてしまうため、\\自分の感情が正しく受け取る相手に伝えられるとは限らない。\end{tabular} \\ \hline
客観 & \begin{tabular}[l]{@{}l@{}}急いで返信する必要はないと感じると面倒と感じ、特に返信が\\「不必要」と感じたり「催促されている」と感じる際にLINEを使用する\\ことで精神的に疲れを感じてしまうのではないかと考えられる。\end{tabular} \\ \hline
客観 & \begin{tabular}[l]{@{}l@{}}すぐに返信が来ると焦ってしまう、返信する立場のことを考えていない\\ように感じ、余計に返信をする気をなくしてしまうという回答もあった。\end{tabular} \\ \hline
客観 & \begin{tabular}[l]{@{}l@{}}このことにより、返信の催促をされ、受け取った側も精神的に疲れを\\感じLINE使用自体が面倒と感じてしまうようになると考えられる。\end{tabular} \\ \hline
\end{tabular}
\label{ambiguous-sbj-obj}
\end{table}

\begin{table}[H]
\centering
\caption{主観客観分類のためのプロンプト(例の組数=3のとき)}
\small % \footnotesize
\begin{tabular}{|l|}
\hline
\multicolumn{1}{|c|}{プロンプト} \\ \hline
\begin{tabular}[c]{@{}l@{}} 
        \#\#\# 役割 \#\#\#\\
        あなた(GPT)を,「大学初年次における日本語文章教育のエキスパート」とします.\\
        \\
        \#\#\# 指示 \#\#\#\\
        「判定する文章」の中の文章それぞれを,主観的であるか客観的であるかを判定してくだ\\さい.\\文章は1文ずつ「。」で区切って提示するので,それぞれについて判定をしてください.\\出力は「フォーマット」に従ってください.「例」では判定理由は省略しています.\\
        <判定理由>は1~2文程度で記述してください.\\
        \\
        \#\#\# 例 \#\#\#\\
        レシートをポケットに入れたまま選択してしまい、洗濯機の掃除に苦労したことがある。\\
        主観的\\
        \\
        風邪を引いてしまい、追試を受けなければならなくなった。\\
        主観的\\
        \\
        病院で処方された薬を飲み忘れてしまったことがある。\\
        主観的\\
        \\
        提出日時に一秒でも遅れてしまうと提出できないようシステム設計されている。\\
        客観的\\
        \\
        Zoomの場合受講生数が多くなると、一画面に収まらなくなり、複数の画面へと分割\\されてしまう。\\
        客観的\\
        \\
        様々な事情により孤独に陥ってしまった人への自治体の支援が十分行き届いていない\\のが現状である。\\
        客観的\\
        \\
        \#\#\# フォーマット \#\#\#\\
        <判定した文章>\\
        <判定結果>\\
        % <判定理由>\\
        \\
        \#\#\# 判定する文章 \#\#\#\\
        「いいね」の数にこだわることは無意味だとはわかっていても、やはり「いいね」の数は\\気になってしまう。\\ \\
        SNSに来たメッセージには「早く返さなければ」と思うため、メッセージを確認する\\ために常にスマホを見てしまうという傾向がある。\\ \\
        スマホ利用実態調査の実験中、「スマホを操作してはならない」という抑圧される\\苦しさに何度も陥ってしまうことがあったが、それでは脱依存を試みることは難しいこと\\も体験した。\\

\end{tabular}   \\ \hline

\end{tabular}
\label{prompt-spokenorwritten}
\end{table}


検証は、以下の項目で行った。
\begin{itemize}
    \item[1] 主観または客観に分類されている文章を、LLMに主観または客観に分類させる
    \item[2] 主観または客観に分類されている文章を、LLMに話しことばまたは書きことばに分類させる
\end{itemize}
話しことばまたは書きことばへの分類は、先行研究の仮定を採用している。以上の検証方法を、プロンプトの組数を0, 1, 2, 7, 13件に分けて行った。

\subsection{主観/客観への分類の結果}

