\chapter{ChatGPTを用いた話しことば検出 \label{c7}}

本章では,LLMを用いた話しことばを検出させる実験について述べる.LLMが持つ話しことばの知識を本システムの話しことばルールを比較し,
\begin{comment}
書きことばリストは、話しことばチェッカーで検出しない表現をもとに選んでいる。
それでもchatGPTが出力する場合は、チェッカーの癖もあり得る。← これは要相談
\end{comment}
\section{7.1 \label{c7s1}}
ChatGPTに

\subsection{検証方法}
プロンプトの構成は,ChatGPTに役割を与える「役割」,取り組んでほしいタスクを示した「指示」,「話しことばの具体例」,「書きことばリスト」,「学生のレポート本文(検出をさせる文章)」の5段構成である.

\begin{table}[H]
\centering
\caption{話しことば検出のためのプロンプト(前半部)}
\small % \footnotesize
\begin{tabular}{|l|}
\hline
\multicolumn{1}{|c|}{プロンプト} \\ \hline
\begin{tabular}[c]{@{}l@{}} 
\#\#\# 役割 \#\#\# \\
あなた(GPT)を,「大学初年次における日本語文章教育のエキスパート」と\\します。\\
\\
\#\#\# 指示 \#\#\# \\
以下の<話しことば具体例>を参考に、\#\#\# 学生のレポート課題の文章 \#\#\#の\\
文章から話しことばにあたる単語をすべて検出してください。\\
出力は\#\#\# フォーマット \#\#\#に従ってください。
なお、公的文書や学術論文、技術文書などで\\使用されるフォーマルな表現を書きことばとします。\\
敬語や丁寧語の文章や日常生活で出てくる文章で使用される表現を話しことばとします。\\
話しことばの具体例は、以下の<話しことば具体例>に載せています。\\
ただし、以下の<書きことばリスト>は話しことばではないので、検出しないでください。\\
<話しことば具体例>\\
1. (~して)いて\\
2. うまく\\
3. かもしれない\\
4. こういった\\
5. (~もある)し\\
6. (~)して\\
7. そんな\\
8. たくさん\\
9. たら\\
10. ていて\\
11. てしまう\\
12. とても\\
13. どんな\\
14. なので\\
15. ので\\
16. (文末の)ます。\\
17. わからない\\
18. 思う\\
19. 私\\
20. 色々\\
21. 素晴らしい\\
22. 分からない\\

\end{tabular}   \\ \hline

\end{tabular}
\label{prompt-detectspoken-api}
\end{table}
\begin{table}[H]
\centering
\caption{話しことば検出のためのプロンプト(後半部)}
\small % \footnotesize
\begin{tabular}{|l|}
\hline
\multicolumn{1}{|c|}{プロンプト} \\ \hline
\begin{tabular}[c]{@{}l@{}} 
<書きことばリスト>\\
1. 考える\\
2. 考えられる\\
3. 使う\\
4. 使われている\\
5. だろう\\
6. ている\\
7. このように\\
8. である\\
9. わかる\\
10. だろうか\\
11. できる\\
12. できるだろうか\\
13. のではないだろうか\\
14. ではないだろうか\\
15. ためである\\
16. つまり\\
17. そもそも\\
18. このような\\
19. どのような\\
20. この\\
21. その\\
22. あの\\
23. どの\\
24. 様々な\\
25. ならば\\
26. 使っている\\
27. この場合\\
28. 学ぶべき\\
29. そんな\\
30. 良い\\
31. 良いだろう\\
32. 役に立つ\\
33. それだけでは無い\\
34. かなり\\
\\
\#\#\# 学生のレポート課題の文章 \#\#\#\\
(レポート本文)\\
\end{tabular}   \\ \hline

\end{tabular}
\label{prompt-sdetectspoken-klist}
\end{table}

書きことばリストは,ChatGPTが検出する「話しことばとされるもの」の中で,明らかに話しことばではないものを選択し,リストにあげたものである.

\section{7.2 \label{c7s2}}
