\chapter{序論\label{c1}}
\pagenumbering{arabic}

\section{背景}
現在、レポートや論文の書き方をはじめとする文章作成技法に関する講義が開講され、様々な教育機関において文章作成能力の向上が求められている。一方で、文章作成技法に関する講義の指導方法は各教育機関や教員によって異なり、模範となる指導方法は確立されていない。これは大学についても同様である。
レポートや論文は、学術的文章に適した表現で記述することが求められる。しかし、学術表現に従ったレポートを作成できない学生が多く存在し、特に、文章中に話しことばが混在することが問題視されている。これらの課題に対し、本研究チームは、話しことばの情報を集約割いたデータベースを活用し、話しことばカテゴリに基づく添削を行う大学初年次向け文章添削システム「話しことばチェッカー」に関する研究を行ってきた。

日本語の話しことばや書きことばは、文章内の単語や、単語同士の関

\section{目的}
本研究では,人間の学びと人工知能の学習ロジックの類似性の観点に基づき,教科としての国語における人間の学習活動を人工知能に再現させ,人工知能の日本語読解能力が向上するかを明らかにすることを目的とする.
文章中の単語を穴埋めする問題を解かせ,正答数と正答した品詞の種類を用いて,本研究で作成したモデルの評価を行う.
小学校全学年の教科書に掲載されている文章を学習データとして,オリジナルの事前学習モデルを構築する.
既存の事前学習済みモデルと本研究で作成したモデルとの差異を分析し,将来的には国語の文章問題の自動添削への応用を目指す.

\section{構成}
本論文の構成について述べる.
第2章では,関連研究について触れ、本研究の位置づけについて述べる.
第3章では,本研究で使用した基盤技術について述べる.特に,研究の要になる BERT について述べる.
第4章では,作成した事前学習モデルについて述べ,既存の事前学習モデルを比較するための評価の方法について述べる.
第5章では,前章の実験の結果をもとにした考察について述べる.
第6章では,実験結果および考察を踏まえた結論と今後の展望について述べる.
