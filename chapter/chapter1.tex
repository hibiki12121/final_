\chapter{序論\label{c1}}
\pagenumbering{arabic}

\section{背景}
現在,レポートや論文の書き方をはじめとする文章作成技法に関する講義が開講され,様々な教育機関において文章作成能力の向上が求められている.一方で,文章作成技法に関する講義の指導方法は各教育機関や教員によって異なり,模範となる指導方法は確立されていない.これは大学についても同様である.
レポートや論文は,学術的文章に適した表現で記述することが求められる.しかし,学術表現に従ったレポートを作成できない学生が多く存在し,特に,文章中に話しことばが混在することが問題視されている.これらの課題に対し,本研究チームは,話しことばの情報を集約割いたデータベースを活用し,話しことばカテゴリに基づく添削を行う大学初年次向け文章添削システム「話しことばチェッカー」に関する研究を行ってきた.

ある表現が話しことばであるか書きことばであるかが採点者によって解釈が分かれる可能性のある曖昧な表現が存在し、文章作成において一貫した指導が難しいとされている。その要因として、あいまいな表現が,話しことばであるか書きことばであるかが文脈の中でどのように用いられているかによって決まる表現が存在することが挙げられる。一方で本システムでは、ルールベースのアルゴリズムに基づく話しことば検出を行うため、正しい検出結果を示すことができないという課題がある。以上のような文脈によって話しことばとも書きことばともとれる表現が存在することがわかっており、本研究チームではこの表現を「グレーゾーン」と定義している。

近年発達してきた機械学習の手法を用いることによって文章内の単語の位置関係以外にも,文脈を考慮した単語の位置関係の把握も行えるようになってきた.また,インターネットの発展とともに,文章や画像のデータセットを無償で利用できることも可能になってきている.先行研究では,グレーゾーンを含む文章が主観的であれば,そのグレーゾーンは話しことば,客観的であれば書きことばに分類できるという仮定のもと,インターネットから収集したグレーゾーンを含む文章を学習させ,グレーゾーンの判別が可能な機械学習モデルの構築を試みた.

しかし,学習データや検証データの確保については,手作業による事例収集やラベル付けについても,専門家が個々のデータを見て判断する必要があるなどといった問題がある.すべてのグレーゾーンの判別が可能な機械学習モデルを構築するために各々のグレーゾーン用のデータセットを作成することは現実的ではない.

\section{目的}


\section{構成}
本論文の構成について述べる.
