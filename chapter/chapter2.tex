\chapter{先行研究と位置づけ\label{c2}}

\section{話しことばチェッカーの開発と実証評価 \label{c2s1}}
「話しことばチェッカーの開発と実証評価」は本研究チームの山下による研究である.学生が提出したレポートに含まれる話しことばの個数を,話しことばチェッカーによる判定を受ける前後でどの程度変化したかの調査が行われている.今後の課題として,システムの誤検出およびパターンマッチによる検出では判断しきれない表現への対応が挙げられており,機械学習を活用した文脈判断による話しことば検出の必要があると述べられている.パターンマッチによる検出では判断しきれない曖昧な表現の例として「~てしまう」が挙げられ,その表現が出現する文章の主語が一人称であれば主観的,三人称であれば客観的となり,機械学習を用いることでそれぞれ話しことば,書きことばに分類できる可能性が示唆されている.

\section{学生のレポートにおける話し言葉とその出現傾向 \label{c2s2}}
「学生のレポートにおける話し言葉とその出現傾向」は本研究チームの山下による研究である.レポートの書き方に関する書籍で取り扱われる話しことばについてまとめ,「学術文章作法I」で提出された学生のレポートから抽出した話しことばについて分析している.ある表現について,話しことばか書きことばかの判断が添削者により異なる表現が存在することや,話しことばをどのように書きことばに書き改めるべきかの判断が難しい事例が存在することなどが示されている.具体例として,「~から」「~ので」について,それぞれを「~ために」に書き改めるよう指示している場合や,「~から」を「~ので」への置き換えを許容する場合が存在する.本研究チームでは,文章に含まれる話しことばを検出する Web システムである話しことばチェッカーの研究および開発を行っているが,このような曖昧な表現について柔軟に対応ができないという課題がある.

\section{学生のレポートにおける話し言葉とその出現傾向 \label{c2s3}}
「学生のレポートにおける話し言葉とその出現傾向」\cite{ai-checker}は川越による研究である.\ref{c2s1}節で示唆されたグレーゾーンを含む文章の主語が一人称であれば主観的,三人称であれば客観的となり,機械学習を用いてそれぞれ話しことば,書きことばに分類できるという観点に基づき,本研究チームが定義したグレーゾーンの一つである「てしまう」に着目し,機械学習が有効かを調査した.機械学習モデルであるBERTを用い,Amazonレビュー文章を主観的な文章,論文データセットを客観的な文章と定めたデータセットでファインチューニングを行った.ファインチューニング後のモデルの正解率は77\%となり,グレーゾーンの話しことば・書きことばの判別に特化したAI話しことばチェッカーの実現可能性が示唆された.一方で,今回は「てしまう」のみの結果であり,グレーゾーンは他にも指定された語句が存在することから,「てしまう」以外のグレーゾーン表現を含む文章の主観・客観を判定できるより汎用的なモデルの構築を課題として残している.

% 日本語の話しことばや書きことばは,文章内の単語や,単語同士の関
