\chapter{第2章予定地\label{c2}}

\section{先行研究 \label{c2s1}}
「話しことばチェッカーの開発と実証評価」は本研究チームの山下による研究である.レポートに含まれる話しことばの個数を,話しことばチェッカーによる判定を受ける前後でどのように変化したかについて述べられている.今後の課題として,システムの誤検出およびパターンマッチでは判断しきれない表現への対応が挙げられており,機械学習を活用した文脈判断による話しことば検出の必要があると述べられている.パターンマッチでは判断しきれない曖昧な表現の例として「~てしまう」が挙げられ,その表現が出現する文章の主語が一人称であれば主観的,三人称であれば客観的となり,機械学習を用いることでそれぞれ話しことば,書きことばに分類できる可能性が示唆されている.

「学生のレポートにおける話し言葉とその出現傾向」は本研究チームの山下による研究である.レポートの書き方に関する書籍で取り扱われる話しことばについてまとめ,「学術文章作法I」で提出された学生のレポートから抽出した話しことばについて分析している.ある表現について,話しことばか書きことばかの判断が添削者により異なる表現が存在することや,話しことばをどのように書きことばに書き改めるべきかの判断が難しい事例が存在することなどが示されている.具体例として,「~から」「~ので」について,それぞれを「~ために」に書き改めるよう指示している場合や,「~から」を「~ので」への置き換えを許容する場合が存在する.本研究チームでは,文章に含まれる話しことばを検出する Web システムである話しことばチェッカーの研究および開発を行っているが,このような曖昧な表現について柔軟に対応ができないという課題がある.