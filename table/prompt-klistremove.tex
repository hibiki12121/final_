\begin{table}[H]
\centering
\caption{書きことばリスト除去のためのプロンプト}
\small % \footnotesize
\begin{tabular}{|l|}
\hline
\multicolumn{1}{|c|}{プロンプト} \\ \hline
\begin{tabular}[c]{@{}l@{}} 
\#\#\# 指示 \#\#\# \\
以下の<書きことばリスト>の1から34までの表現は書きことばとします。\\
<書きことばリスト>の言葉が\#\#\# 表現リスト \#\#\#に含まれている場合、\\ \#\#\# 表現リスト \#\#\#から除外して出力してください。\\
<書きことばリスト>\\
1. 考える\\
2. 考えられる\\
3. 使う\\
4. 使われている\\
5. だろう\\
6. ている\\
7. このように\\
8. である\\
9. わかる\\
10. だろうか\\
11. できる\\
12. できるだろうか\\
13. のではないだろうか\\
14. ではないだろうか\\
15. ためである\\
16. つまり\\
17. そもそも\\
18. このような\\
19. どのような\\
20. この\\
21. その\\
22. あの\\
23. どの\\
24. 様々な\\
25. ならば\\
26. 使っている\\
27. この場合\\
28. 学ぶべき\\
29. そんな\\
30. 良い\\
31. 良いだろう\\
32. 役に立つ\\
33. それだけでは無い\\
34. かなり\\

\#\#\# 表現リスト \#\#\# \\
(1度目の出力で得られた「話しことばとされたもの」)
\end{tabular}   \\ \hline

\end{tabular}
\label{prompt-klistremove}
\end{table}
% 使用箇所: c7s4