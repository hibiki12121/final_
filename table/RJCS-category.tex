\begin{table}[H]
\centering
\caption{話しことばカテゴリ}
\begin{tabular}{|c|l|l|l|}
\hline
\multicolumn{1}{|c|}{番号} & \multicolumn{1}{c|}{名称} & \multicolumn{1}{c|}{説明} & \multicolumn{1}{c|}{例} \\ \hline
1 & INDEPENDENCE & 対象の単語がある           & \begin{tabular}[c]{@{}l@{}}当たり前、\\あんまり\end{tabular} \\ \hline
2 & PREFIX       & \begin{tabular}[c]{@{}l@{}}対象の単語およびその直前に\\ 特定の品詞の単語がある\end{tabular} & \begin{tabular}[c]{@{}l@{}}名詞+して、\\動詞+して\end{tabular} \\ \hline
3 & SUFFIX       & \begin{tabular}[c]{@{}l@{}}品詞の組み合わせを必要とせず、\\なおかつ複数の単語から成り立つ\end{tabular} & \begin{tabular}[c]{@{}l@{}}くせ+に、\\ かも+しれ+ない\end{tabular} \\ \hline
4 & COLLOCATION  & \begin{tabular}[c]{@{}l@{}}対象の単語およびその単語と\\同じ文章に特定の単語がある\end{tabular} & \begin{tabular}[c]{@{}l@{}}一番+形容詞\\ どうしても+たい \end{tabular} \\ \hline
5 & OTHER   & \begin{tabular}[c]{@{}l@{}}文法的誤り、若者的な言葉遣いなど、\\上記4つ以外に当てはまるもの\end{tabular}  & 全然+違う \\ \hline
\end{tabular}
\label{RJCS-category}
\end{table}

% 使用箇所: c3s2