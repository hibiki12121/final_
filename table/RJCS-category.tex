\begin{table}[H]
\centering
\caption{あああ}
\begin{tabular}{|c|l|l|l|}
\hline
\multicolumn{1}{|c|}{カテゴリ} & \multicolumn{1}{c|}{名称} & \multicolumn{1}{c|}{説明} & \multicolumn{1}{c|}{例} \\ \hline
1                                & INDEPENDENCE              & 対象の単語がある           & \\ \hline
2                                & PREFIX                    & \begin{tabular}[c]{@{}l@{}}対象の単語およびその直前に\\ 特定の品詞の単語がある\end{tabular} &  \\ \hline
3                                & SUFFIX                    &                          & \\ \hline
4                                & COLLOCATION               &                          & \\ \hline
5                                & OTHER                     &                          & \\ \hline
\end{tabular}
\label{RJCS-category}
\end{table}