\begin{table}[H]
\centering
\caption{Few-shot Promptingを用いた話しことば検出のためのプロンプト}
\small % \footnotesize
\begin{tabular}{|l|}
\hline
\multicolumn{1}{|c|}{プロンプト} \\ \hline
\begin{tabular}[c]{@{}l@{}} 
\#\#\# 役割 \#\#\#\\
あなた(GPT)を,「大学初年次における日本語文章教育のエキスパート」\\とします。\\
\\
\#\#\# 指示 \#\#\#\\
与えられた文章から、話しことばである表現を抽出してください。\\
ここでの話しことばの定義は「学術表現として使用することが適切\\ではない表現」です。\\
話しことばを抽出したとき、以下の情報を出力してください。\\
1. 抽出した話しことば\\
\\
\#\#\# 例 \#\#\#\\
Q. 以下の文章から、話しことばである表現を抽出してください。\\
日本人はお米を食べるのが当たり前のくせにパンばかり食べる。\\この件の対応策の策定は政府がやらなければならないとされており、\\政府は検討して要望を提出すると述べている。\\
A. 抽出した話し言葉は以下の通りです。\\
   - 抽出した話しことば1 : 当たり前\\
   - 抽出した話しことば2 : くせに\\
   - 抽出した話しことば3 : やらなければならない\\
   - 抽出した話しことば4 : 検討して\\
\\
(中略)\\
\\
Q. 以下の文章から、話しことばである表現を抽出してください。\\
自ら走ることもあるので\\
A. 抽出した話し言葉は以下の通りです。\\
   - 抽出した話しことば1 : ので\\
\\
\#\#\# 実践 \#\#\# \\
Q. 以下の文章から、話しことばである表現を抽出してください。\\
(レポート本文)
\end{tabular}   \\ \hline
\end{tabular}
\label{prompt-detectspoken-few10}
\end{table}