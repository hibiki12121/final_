\begin{table}[H]
\centering
\caption{話しことば検出のためのプロンプト(前半部)}
\small % \footnotesize
\begin{tabular}{|l|}
\hline
\multicolumn{1}{|c|}{プロンプト} \\ \hline
\begin{tabular}[c]{@{}l@{}} 
\#\#\# 役割 \#\#\# \\
あなた(GPT)を,「大学初年次における日本語文章教育のエキスパート」と\\します。\\
\\
\#\#\# 指示 \#\#\# \\
以下の<話しことば具体例>を参考に、\#\#\# 学生のレポート課題の文章 \#\#\#の\\
文章から話しことばにあたる単語をすべて検出してください。\\
出力は\#\#\# フォーマット \#\#\#に従ってください。
なお、公的文書や学術論文、技術文書などで\\使用されるフォーマルな表現を書きことばとします。\\
敬語や丁寧語の文章や日常生活で出てくる文章で使用される表現を話しことばとします。\\
話しことばの具体例は、以下の<話しことば具体例>に載せています。\\
ただし、以下の<書きことばリスト>は話しことばではないので、検出しないでください。\\
<話しことば具体例>\\
1. (~して)いて\\
2. うまく\\
3. かもしれない\\
4. こういった\\
5. (~もある)し\\
6. (~)して\\
7. そんな\\
8. たくさん\\
9. たら\\
10. ていて\\
11. てしまう\\
12. とても\\
13. どんな\\
14. なので\\
15. ので\\
16. (文末の)ます。\\
17. わからない\\
18. 思う\\
19. 私\\
20. 色々\\
21. 素晴らしい\\
22. 分からない\\

\end{tabular}   \\ \hline

\end{tabular}
\label{prompt-detectspoken-api}
\end{table}