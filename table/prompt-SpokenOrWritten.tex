\begin{table}[H]
\centering
\caption{主観客観分類のためのプロンプト(例の組数=3のとき)}
\small % \footnotesize
\begin{tabular}{|l|}
\hline
\multicolumn{1}{|c|}{プロンプト} \\ \hline
\begin{tabular}[c]{@{}l@{}} 
        \#\#\# 役割 \#\#\#\\
        あなた(GPT)を,「大学初年次における日本語文章教育のエキスパート」とします.\\
        \\
        \#\#\# 指示 \#\#\#\\
        「判定する文章」の中の文章それぞれを,主観的であるか客観的であるかを判定してくだ\\さい.\\文章は1文ずつ「。」で区切って提示するので,それぞれについて判定をしてください.\\出力は「フォーマット」に従ってください.「例」では判定理由は省略しています.\\
        <判定理由>は1~2文程度で記述してください.\\
        \\
        \#\#\# 例 \#\#\#\\
        レシートをポケットに入れたまま選択してしまい、洗濯機の掃除に苦労したことがある。\\
        主観的\\
        \\
        風邪を引いてしまい、追試を受けなければならなくなった。\\
        主観的\\
        \\
        病院で処方された薬を飲み忘れてしまったことがある。\\
        主観的\\
        \\
        提出日時に一秒でも遅れてしまうと提出できないようシステム設計されている。\\
        客観的\\
        \\
        Zoomの場合受講生数が多くなると、一画面に収まらなくなり、複数の画面へと分割\\されてしまう。\\
        客観的\\
        \\
        様々な事情により孤独に陥ってしまった人への自治体の支援が十分行き届いていない\\のが現状である。\\
        客観的\\
        \\
        \#\#\# フォーマット \#\#\#\\
        <判定した文章>\\
        <判定結果>\\
        <判定理由>\\
        \\
        \#\#\# 判定する文章 \#\#\#\\
        「いいね」の数にこだわることは無意味だとはわかっていても、やはり「いいね」の数は\\気になってしまう。\\ \\
        SNSに来たメッセージには「早く返さなければ」と思うため、メッセージを確認する\\ために常にスマホを見てしまうという傾向がある。\\ \\
        スマホ利用実態調査の実験中、「スマホを操作してはならない」という抑圧される\\苦しさに何度も陥ってしまうことがあったが、それでは脱依存を試みることは難しいこと\\も体験した。\\

\end{tabular} &  \\ \hline

\end{tabular}
\label{prompt-spokenorwritten}
\end{table}
