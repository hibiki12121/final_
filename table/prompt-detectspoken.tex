
\begin{table}[H]
\centering
\caption{主観客観分類のためのプロンプト(例の組数=3のとき)}
\small % \footnotesize
\begin{tabular}{|l|}
\hline
\multicolumn{1}{|c|}{プロンプト} \\ \hline
\begin{tabular}[c]{@{}l@{}} 
理工学分野を専攻する大学1~2年生が執筆したレポートを添削するという想定で、「話しことば」の検出をしてください。\\
ここでの話しことばとは、口語的な表現に加え、レポートとして使用することが適切でない表現も含めることとします。 \\
\\
【検出対象の文章】
(レポート本文)
\\
出力は、あなたが「話しことば」であると判断した表現を、箇条書きで示してください。\\表現の単位は単語ごとです。\\また、検出した話しことばを適切な表現に修正するときに具体的な修正案を提示してください。: \\
\\
【検出した話しことば】\\
\\
1. 検出した話しことば① \\指摘理由:(話しことば①を検出した理由を示す) \\修正案:(話しことば①の修正例を示す: (「修正例①」、「修正例②」、……)) 
2. 検出した話しことば② \\指摘理由:(話しことば②を検出した理由を示す) \\修正案:(話しことば②の修正例を示す: (「修正例①」、「修正例②」、……))

\end{tabular}   \\ \hline

\end{tabular}
\label{prompt-spokendetect}
\end{table}