\begin{table}[H]
\centering
\caption{データセットより抜粋}
\begin{tabular}{|c|l|}
\hline
\multicolumn{1}{|c|}{主観/客観} & \multicolumn{1}{c|}{例文} \\ \hline
主観 & \begin{tabular}[l]{@{}l@{}}以前はゴルフのプレー中にコンタクトが取れてしまうことがあり、\\いつも目のことを気にしていなければならなかった。\end{tabular} \\ \hline
主観 & \begin{tabular}[l]{@{}l@{}}無神経な発言にいちいち腹が立ってしまう自分も大人気ないと思う。\end{tabular} \\ \hline
主観 & \begin{tabular}[l]{@{}l@{}}あせって解約して、思わぬ損をしてしまった。\end{tabular} \\ \hline
主観 & \begin{tabular}[l]{@{}l@{}}言うつもりではなかったのに口をすべらせてしまうことがある。\end{tabular} \\ \hline
客観 & \begin{tabular}[l]{@{}l@{}}絵文字・顔文字は機種間によって表示方法に差異が生じてしまうため、\\自分の感情が正しく受け取る相手に伝えられるとは限らない。\end{tabular} \\ \hline
客観 & \begin{tabular}[l]{@{}l@{}}急いで返信する必要はないと感じると面倒と感じ、特に返信が\\「不必要」と感じたり「催促されている」と感じる際にLINEを使用する\\ことで精神的に疲れを感じてしまうのではないかと考えられる。\end{tabular} \\ \hline
客観 & \begin{tabular}[l]{@{}l@{}}すぐに返信が来ると焦ってしまう、返信する立場のことを考えていない\\ように感じ、余計に返信をする気をなくしてしまうという回答もあった。\end{tabular} \\ \hline
客観 & \begin{tabular}[l]{@{}l@{}}このことにより、返信の催促をされ、受け取った側も精神的に疲れを\\感じLINE使用自体が面倒と感じてしまうようになると考えられる。\end{tabular} \\ \hline
\end{tabular}
\label{ambiguous-sbj-obj}
\end{table}

% 使用箇所: c6s2