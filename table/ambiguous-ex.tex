\begin{table}[H]
\centering
\caption{グレーゾーンの例}
\begin{tabular}{|l|l|}
\hline
\multicolumn{1}{|c|}{表現} & \multicolumn{1}{c|}{例文} \\ \hline
てしまう   & \begin{tabular}[l]{@{}l@{}}アップライトピアノだと弾いている指が隠れてしまう\end{tabular} \\ \hline
残念      & \begin{tabular}[l]{@{}l@{}}楽しみにしていたイベントが中止になり残念だった\end{tabular} \\ \hline
面倒くさい & \begin{tabular}[l]{@{}l@{}}私生活では話すことを面倒くさがらずに、\\聞かれたことを真摯に答え説明する力をつけるように努力する\end{tabular} \\ \hline
厳しい     & \begin{tabular}[l]{@{}l@{}}ノルマを達成しなければ昇進は厳しい\end{tabular} \\ \hline
感じる     & \begin{tabular}[l]{@{}l@{}}人それぞれ大丈夫と感じることは異なっている\end{tabular} \\ \hline
大切      & \begin{tabular}[l]{@{}l@{}}社会で活躍できる人材を育成することも大切であると考える\end{tabular} \\ \hline
つらい     & \begin{tabular}[l]{@{}l@{}}芸能界は華やかに見えても、肉体的にも精神的にもつらい仕事がある\end{tabular} \\ \hline
\end{tabular}
\label{ambiguous-ex}
\end{table}