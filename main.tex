\documentclass[12pt, a4paper]{ujreport} % a4jでもOKっぽい

%% ----- packages ----- %%
% --- 図表 --- %
\usepackage{plautopatch}
\usepackage[dvipdfmx]{graphicx, xcolor, pict2e} % geometory: 余白設定 graphicx: 画像挿入 xcolor: 文字の色付け pict2e: 図が描ける
\usepackage{here} % 図の配置を指定できる
\usepackage{float} % here と同じらしい
\usepackage{longtable} % 表
\usepackage{colortbl} % 表に色を付ける
\usepackage{tabularx} % 表組みの設定
\usepackage{enumerate} % 番号付き箇条書き
\usepackage{lscape} % 図表回転
\usepackage{multirow} % 表のセルに連続して同じ値を書くときにまとめて書ける(行)
\usepackage{diagbox} % 表ヘッダ作成
\usepackage{multicol} % 表のセルに連続して同じ値を書くときにまとめて書ける(列)
\usepackage{wrapfig} % 図と文章を並べて書ける
\usepackage{hhline}
% \usepackage{graphicx}

% --- 文字 --- %
\usepackage{bm} % 太字の斜体
\usepackage{underscore} % アンダーバー"_"を表示できる
\usepackage{plext} % 縦書きを横書きで書ける
\usepackage{fancyhdr} % ヘッダとフッタを編集できる
\usepackage{ulem} % 下線が引ける
\usepackage{cite} % 参考文献リスト"bibitem"の参照
\usepackage{times} % フォント "times" に指定
\usepackage{mathptmx} % 数式も含めてフォント "times" に指定
\usepackage{amsmath,amssymb,amsfonts} % 数式をきれいに書ける。amsfontsはフォントを指定しているが、mathptmxでフォント指定されているため、機能していない。
\usepackage{algorithmic} % アルゴリズム(疑似コード)を書ける
\usepackage{textcomp} % 特殊文字を書ける
\usepackage{comment} % コメントアウト
\usepackage{url} % url挿入
%\usepackage[ipaex]{pxchfon} %文書内の和文フォントの指定(ipaexフォントが指定されている, 無くても大丈夫そう)
%\usepackage[uplatex, deluxe]{otf}
% \usepackage{color}
% \usepackage{xcolor} % 文字の色付けができる(color の上位互換?)

%% --- ページ設定 --- %%
% \usepackage[top=30truemm,bottom=30truemm,left=30truemm,right=20truemm]{geometry} % 余白設定

% (すぐ上の行を使わないとき用)
\makeatletter
% \setlength{\voffset}{0.5truecm} % old 上余白 (1)
% \setlength{\oddsidemargin}{7.0truemm} % old 左余白 (3): 25.4mm + 7.0mm = 32.4mm
\setlength{\oddsidemargin}{4.6truemm} % 左余白: 25.4mm + 4.6mm = 30mm, 13pt
% \setlength{\evensidemargin}{-5.5truemm} % old 偶数頁左余白 (3): 25.4 - 5.5mm = 19.9mm, jreportなので不要?
% \setlength{\topmargin}{-1truecm} % old (4)
\setlength{\topmargin}{-23pt} % -12pt+(-11pt) = -23pt, 8.1mm (4)
% \setlength{\headsep}{2truecm} % old 上余白 (6)
\setlength{\headsep}{25pt} % 8.81mm, 12pt+14pt=25pt (6), 上余白30mmになるように設定した
% \setlength{\textheight}{22truecm} % old (7)
\setlength{\textheight}{672pt} % 845pt-85pt*2, 237truemm (7), 下余白30mmになるように設定した
% \setlength{\textwidth}{15truecm} % old (8)
\setlength{\textwidth}{441pt} % (8) 155truemm, 右余白25mmになるように設定した
% \setlength{\footskip}{25truemm} % old (11)
\setlength{\footskip}{30pt} % 10.5pt+19.5pt=30pt, 10.6mm(11), ページ数の位置
% 1pt = 0.35mm
\makeatother


%% ----- main ----- %%
\begin{document}

\pagenumbering{roman}
\title{BERTを用いた小学生の国語学習再現に関する研究\\
A Study on the Reproduction of Japanese Language Learning for Elementary School Students Using BERT}
\author{指導教員: 小松川 浩\\
公立千歳科学技術大学 理工学部\\
情報システム工学科 小松川・山川・深町研究室\\
令和3年度\\
b2181920 新井田 響}
\date{\today}
\maketitle

\tableofcontents
\thispagestyle{empty}

\chapter{序論\label{c1}}
\pagenumbering{arabic}

\section{背景}
現在,レポートや論文の書き方をはじめとする文章作成技法に関する講義が開講され,様々な教育機関において文章作成能力の向上が求められている.一方で,文章作成技法に関する講義の指導方法は各教育機関や教員によって異なり,模範となる指導方法は確立されていない.これは大学についても同様である.
レポートや論文は,学術的文章に適した表現で記述することが求められる.しかし,学術表現に従ったレポートを作成できない学生が多く存在し,特に,文章中に話しことばが混在することが問題視されている.これらの課題に対し,本研究チームは,話しことばの情報を集約割いたデータベースを活用し,話しことばカテゴリに基づく添削を行う大学初年次向け文章添削システム「話しことばチェッカー」に関する研究を行ってきた.

ある表現が話しことばであるか書きことばであるかが採点者によって解釈が分かれる可能性のある曖昧な表現が存在し、文章作成において一貫した指導が難しいとされている。その要因として、あいまいな表現が,話しことばであるか書きことばであるかが文脈の中でどのように用いられているかによって決まる表現が存在することが挙げられる。一方で本システムでは、ルールベースのアルゴリズムに基づく話しことば検出を行うため、正しい検出結果を示すことができないという課題がある。以上のような文脈によって話しことばとも書きことばともとれる表現が存在することがわかっており、本研究チームではこの表現を「グレーゾーン」と定義している。

近年発達してきた機械学習の手法を用いることによって文章内の単語の位置関係以外にも,文脈を考慮した単語の位置関係の把握も行えるようになってきた.また,インターネットの発展とともに,文章や画像のデータセットを無償で利用できることも可能になってきている.先行研究では,グレーゾーンを含む文章が主観的であれば,そのグレーゾーンは話しことば,客観的であれば書きことばに分類できるという仮定のもと,インターネットから収集したグレーゾーンを含む文章を学習させ,グレーゾーンの判別が可能な機械学習モデルの構築を試みた.

しかし,学習データや検証データの確保については,手作業による事例収集やラベル付けについても,専門家が個々のデータを見て判断する必要があるなどといった問題がある.すべてのグレーゾーンの判別が可能な機械学習モデルを構築するために各々のグレーゾーン用のデータセットを作成することは現実的ではない.

\section{目的}


\section{構成}
本論文の構成について述べる.




\end{document}