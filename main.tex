\documentclass[12pt, a4paper]{ujreport} % a4jでもOKっぽい

%% ----- packages ----- %%
% --- 図表 --- %
\usepackage{plautopatch}
\usepackage[dvipdfmx]{graphicx, xcolor, pict2e} % geometory: 余白設定 graphicx: 画像挿入 xcolor: 文字の色付け pict2e: 図が描ける
\usepackage{here} % 図の配置を指定できる
\usepackage{float} % here と同じらしい
\usepackage{longtable} % 表
\usepackage{colortbl} % 表に色を付ける
\usepackage{tabularx} % 表組みの設定
\usepackage{enumerate} % 番号付き箇条書き
\usepackage{lscape} % 図表回転
\usepackage{multirow} % 表のセルに連続して同じ値を書くときにまとめて書ける(行)
\usepackage{diagbox} % 表ヘッダ作成
\usepackage{multicol} % 表のセルに連続して同じ値を書くときにまとめて書ける(列)
\usepackage{wrapfig} % 図と文章を並べて書ける
\usepackage{hhline}
% \usepackage{graphicx}

% --- 文字 --- %
\usepackage{bm} % 太字の斜体
\usepackage{underscore} % アンダーバー"_"を表示できる
\usepackage{plext} % 縦書きを横書きで書ける
\usepackage{fancyhdr} % ヘッダとフッタを編集できる
\usepackage{ulem} % 下線が引ける
\usepackage{cite} % 参考文献リスト"bibitem"の参照
\usepackage{times} % フォント "times" に指定
\usepackage{mathptmx} % 数式も含めてフォント "times" に指定
\usepackage{amsmath,amssymb,amsfonts} % 数式をきれいに書ける。amsfontsはフォントを指定しているが、mathptmxでフォント指定されているため、機能していない。
\usepackage{algorithmic} % アルゴリズム(疑似コード)を書ける
\usepackage{textcomp} % 特殊文字を書ける
\usepackage{comment} % コメントアウト
\usepackage{url} % url挿入
%\usepackage[ipaex]{pxchfon} %文書内の和文フォントの指定(ipaexフォントが指定されている, 無くても大丈夫そう)
%\usepackage[uplatex, deluxe]{otf}

%% --- ページ設定 --- %%
% \usepackage[top=30truemm,bottom=30truemm,left=30truemm,right=20truemm]{geometry} % 余白設定

% (すぐ上の行を使わないとき用)
\makeatletter
% \setlength{\voffset}{0.5truecm} % old 上余白 (1)
% \setlength{\oddsidemargin}{7.0truemm} % old 左余白 (3): 25.4mm + 7.0mm = 32.4mm
\setlength{\oddsidemargin}{4.6truemm} % 左余白: 25.4mm + 4.6mm = 30mm, 13pt
% \setlength{\evensidemargin}{-5.5truemm} % old 偶数頁左余白 (3): 25.4 - 5.5mm = 19.9mm, jreportなので不要?
% \setlength{\topmargin}{-1truecm} % old (4)
\setlength{\topmargin}{-23pt} % -12pt+(-11pt) = -23pt, 8.1mm (4)
% \setlength{\headsep}{2truecm} % old 上余白 (6)
\setlength{\headsep}{25pt} % 8.81mm, 12pt+14pt=25pt (6), 上余白30mmになるように設定した
% \setlength{\textheight}{22truecm} % old (7)
\setlength{\textheight}{672pt} % 845pt-85pt*2, 237truemm (7), 下余白30mmになるように設定した
% \setlength{\textwidth}{15truecm} % old (8)
\setlength{\textwidth}{441pt} % (8) 155truemm, 右余白25mmになるように設定した
% \setlength{\footskip}{25truemm} % old (11)
\setlength{\footskip}{30pt} % 10.5pt+19.5pt=30pt, 10.6mm(11), ページ数の位置
% 1pt = 0.35mm
\makeatother


%% ----- main ----- %%
\begin{document}

\pagenumbering{roman}
\title{
タイトル未定\\
untitled
}
\author{
指導教員: 小松川 浩 \\
公立千歳科学技術大学 理工学部 \\
理工学専攻 博士前期課程2年 \\
小松川・山川・深町研究室 \\
令和5年度 \\
m2220180 新井田 響}
\date{\today}
\maketitle

\tableofcontents
\thispagestyle{empty}

\chapter{序論\label{c1}}
\pagenumbering{arabic}

\section{背景}
現在,レポートや論文の書き方をはじめとする文章作成技法に関する講義が開講され,様々な教育機関において文章作成能力の向上が求められている.一方で,文章作成技法に関する講義の指導方法は各教育機関や教員によって異なり,模範となる指導方法は確立されていない.これは大学についても同様である.
レポートや論文は,学術的文章に適した表現で記述することが求められる.しかし,学術表現に従ったレポートを作成できない学生が多く存在し,特に,文章中に話しことばが混在することが問題視されている.これらの課題に対し,本研究チームは,話しことばの情報を集約割いたデータベースを活用し,話しことばカテゴリに基づく添削を行う大学初年次向け文章添削システム「話しことばチェッカー」に関する研究を行ってきた.


\section{目的}


\section{構成}

 % 背景目的
\chapter{第2章予定地\label{c2}}

\section{先行研究 \label{c2s1}}
「話しことばチェッカーの開発と実証評価」は本研究チームの山下による研究である.学生が提出したレポートに含まれる話しことばの個数を,話しことばチェッカーによる判定を受ける前後でどの程度変化したかの調査が行われている.今後の課題として,システムの誤検出およびパターンマッチによる検出では判断しきれない表現への対応が挙げられており,機械学習を活用した文脈判断による話しことば検出の必要があると述べられている.パターンマッチによる検出では判断しきれない曖昧な表現の例として「~てしまう」が挙げられ,その表現が出現する文章の主語が一人称であれば主観的,三人称であれば客観的となり,機械学習を用いることでそれぞれ話しことば,書きことばに分類できる可能性が示唆されている.

\section{先行研究 \label{c2s2}}
「学生のレポートにおける話し言葉とその出現傾向」は本研究チームの山下による研究である.レポートの書き方に関する書籍で取り扱われる話しことばについてまとめ,「学術文章作法I」で提出された学生のレポートから抽出した話しことばについて分析している.ある表現について,話しことばか書きことばかの判断が添削者により異なる表現が存在することや,話しことばをどのように書きことばに書き改めるべきかの判断が難しい事例が存在することなどが示されている.具体例として,「~から」「~ので」について,それぞれを「~ために」に書き改めるよう指示している場合や,「~から」を「~ので」への置き換えを許容する場合が存在する.本研究チームでは,文章に含まれる話しことばを検出する Web システムである話しことばチェッカーの研究および開発を行っているが,このような曖昧な表現について柔軟に対応ができないという課題がある.

\section{先行研究 \label{c2s3}}
「AI 話しことばチェッカーを想定した機械学習モデリングの研究」は川越による研究である。\ref{c2s1}節で示唆されたグレーゾーンを含む文章の主語が一人称であれば主観的、三人称であれば客観的となり、機械学習を用いてそれぞれ話しことば、書きことばに分類できるという観点に基づき、本研究チームが定義したグレーゾーンの一つである「てしまう」に着目し、機械学習が有効かを調査した。機械学習モデルであるBERTを用い、Amazonレビュー文章を主観的な文章、論文データセットを客観的な文章と定めたデータセットでファインチューニングを行った。ファインチューニング後のモデルの正解率は77\%となり、グレーゾーンの話しことば・書きことばの判別に特化したAI話しことばチェッカーの実現可能性が示唆された。一方で、今回は「てしまう」のみの結果であり、グレーゾーンは他にも指定された語句が存在することから、「てしまう」以外のグレーゾーン表現を含む文章の主観・客観を判定できるより汎用的なモデルの構築を課題として残している。 % 先行研究位置づけ
\chapter{第3章予定地\label{c3}} % 
\chapter{第4章予定地\label{c4}}
\begin{comment}
機械学習
評価指標
など
\end{comment}

\section{4.1\label{c4s1}}
機械学習は、特定のタスクを効率的に処理するため、あるデータが持っているルールやパターンを反復学習を用いて学習し、そこから得られたパターンをもとに予測を行う技術である。2010年代にニューラルネットワークや自然言語処理が登場し、機械学習が再び注目されるようになった。
さらに、ハードウェアの性能の向上およびインターネットの発展により、機械学習に利用できる学習データを容易に収集できるようになったことから、ニューラルネットワークを用いた機械学習が広く用いられるようになった。ニューラルネットワークは、人体の脳の神経細胞であるニューロンおよびそのつながりである神経回路を数理モデルで表現したものである。
ニューラルネットワークが登場する前の機械学習では、データの特徴量などは人間の手によって設定されていたが、特徴量の調整などの調整は機械が自動で行うため、元のデータを与えることで、従来の人間の手による調整よりも高い精度で予測を行うことができる。

機械学習の分野の1つに自然言語処理がある.自然言語とは人間が日常的に使用している言語を指し,これをコンピュータを用いて処理する技術のことである.自然言語処理には大きく4つの段階でテキストデータの処理を行う.

\begin{enumerate}
    \item 形態素解析
    \item 構文解析
    \item 意味解析
    \item 文脈解析
\end{enumerate}

自然言語処理を行う上で,処理の対象となる文書データと,文書に含まれる単語を識別するための辞書が必要となる.自然言語処理分野において課題に挙げられることは,自然言語が本質的に持つ「曖昧さ」が挙げられ,これが文章の解釈が複雑になる要因となっている.これは,単語が持つ意味が複数存在し,その単語が用いられている文章の文脈によって意味が変化するためである.この影響を軽減させるためには,自然言語処理を行うモデルに大量のテキストデータを学習させる必要がある.

\section{二値分類における評価指標 \label{c4s2}}

\subsection{混合行列 \label{c4s2-1}}
機械学習における二値分類の評価指標に,混合行列(Confusion Matrix)がある.二値分類で出力されたクラス分類結果をまとめた行列であり,二値分類タスクを行う機械学習モデルの評価指標として利用される.
 
% 図挿入予定

混合行列左上のTP(True Positive)は,実際のデータが正であるものに対し正と予想されたデータの数である.FP(False Positive)は,実際のデータが正であるものに対し負と予想されたデータである.TN(True Negative)は,実際のデータが負であるものに対し負と予想されたデータである.FN(False Negative)は,実際のデータが負であるものに対し正と予想されたデータの個数である.これら4つの値を用いて,後述する正解率,再現率,適合率,F1値を算出する.

\subsubsection{正解率 \label{c4s2-1a}}
正解率(Accuracy)は,二値分類タスクの評価指標の一つであり,全体の分類結果のうち正答した割合を示す値である.正解率の定義を次式に示す.

$$
Accuracy = \frac{TP+FN}{TP+TN+FP+FN}
$$


\subsubsection{適合率 \label{c4s2-1b}}
適合率は(Precision)は,二値分類タスクにおける評価指標の一つであり,学習モデルが正と予測したもののうち,実際に正であものの割合を示す値である.適合率の定義を次式に示す.

$$
Precision = \frac{TP}{TP+FP}
$$


\subsubsection{再現率 \label{c4s2-1c}}
再現率(Recall)は,二値分類タスクの評価指標の一つれあり,実際のデータに含まれる正クラス全体のうち,学習モデルが正と予測したものの割合を示す値である.再現率の定義を次式に示す.

$$
Recall = \frac{TP}{TP+FN}
$$

\subsubsection{F1値 (F1-score, F-measure) \label{c4s2-1d}}
F1値は,二値分類タスクの一つであり,適合率と再現率の調和平均で算出される値である.適合率と再現率の関係はトレードオフの関係であるため,適合率と再現率の間に差が生じる場合がある.この場合,精度が良いとは一概には判断できない.そのため,これら2つの調和平均を算出し,精度に対し2値のバランスを判断するために用いられる.

$$
F1 = \frac{2 \times Recall \times Precision}{Recall+Precision}
$$


\section{Attention \label{c4s3}}
Attention は,先述の通り,Transformer の中枢を担う仕組みである.
自然言語処理においては,単語の意味を理解するために,文中のどの単語に注目するすべきかを示すスコアである.Query:${Q}$,Key:${K}$,Value:${V}$の3つのベクトルから計算される.
${K}$と${V}$は1対1の組である.
Attentionは,Self-AttentionとSource-Target-Attentionの2種類がある.また,Attentionの算出方法は加法を使う場合と内積を使う場合があるが,Transformerで使うAttentionは内積を使って算出するため,以降のAttentionの計算は内積を使用していることを前提とする.

\subsection{Self-Attention \label{c4s3-1a}}
${Q}$, ${K}$, ${V}$は全て同じデータから得られた値を使用する.
例として「私/は/大学生/です」という文から${Q}$を得たとすると,${K}$, ${V}$も同じ文から取得する.Transformer では Encoder, Decoder の両方に採用され,文の構造や,形態素同士の関係(例文では「私」=「大学生」)を獲得するために使用される.

\subsection{Target-Attention \label{c4s3-1b}}
${Q}$と${K}$, ${V}$は異なるデータから得られた値を使用する(${K}$, ${V}$は同じデータから取得する).
例として「風邪/を/引いた」,「病院/に/行く」という2文があるとき,${Q}$は「風邪/を/引いた」から,${K}$, ${V}$は「病院/に/行く」から取得する.
 Transfomer では Decoder で採用され,「風邪/を/引いた」→「病院/に/行く」という対話の学習に用いられる.
入力文に対応する出力文が出力されるように学習を行う.

以上の Attention を基本とし,Transformers では Scaled Dot-Product Attention と Multi-Head Attentionが実装されている.
その仕組みを図aaa,図aaaに示す.

% % \begin{figure}[H]
% 	\centering
% 	\includegraphics[width=80mm]{image/transformer-multi-head-attention.png}
% 	\caption{Multi-Head Attentionの構造}
% 	\label{mha}
% \end{figure}

% % \begin{figure}[H]
% 	\centering
% 	\includegraphics[width=80mm]{image/transformer-scaled-dot-product-attention.png}
% 	\caption{Scaled Dot-Product Attentionの構造}
% 	\label{sda}
% \end{figure}

% 	${Q}$ ${K}$ ${V}$

\subsubsection{Scaled Dot-Product Attention}
 ${Q}$に対応する${K}$を探し,その${K}$を元にして対応する${V}$を取得する.
まず,${Q}$と${K}$の内積${QK^T}$を取ることで${Q}$に対する${K}$の関連度を算出する.
次に softmax 関数を用いて正規化する.ここで,正規化された値は,Attention の重みであり${Q}$に対応する${K}$の位置を示している.
次に Attention の重みと${V}$の内積を求め,${K}$の位置に対応する${V}$を加重和として取得する.
図 3.5 の数式を式(\ref{attention})に,softmax 関数を式(\ref{softmax})に示す.

\begin{equation}
    \mbox{Attention}(Q,K,V) = \mbox{softmax}\left( \frac{QK^T}{\sqrt{d_k}}\right)
    \label{attention}
\end{equation}

\begin{equation}
    \label{softmax}
    \frac{\exp(a_i)}{\sum_{j=1}^{n}\exp(a_j)} \quad(i=1,...,n)
\end{equation}

${QK^T}$の値は次元数に比例して大きくなり,勾配は小さくなってしまう.
そこで式\ref{attention}では${QK^T}$を${\sqrt{d_k}}$で割ることで${QK^T}$の値の増大を防いでいる.
ここで,${d_k}$は${Q}$の次元数を後述する Multi-Head Attention の Head 数で割った値である.

\begin{equation}
    \label{dk}
    d_k = \frac{Q\mbox{の次元数}}{\mbox{Multi-Head Attention のHead数}}
\end{equation}

\subsubsection{Multi-Head Attention}
Multi-Head Attention は,Scaled Dot-Product Attention を1つの Head として,複数の Head を並列で処理する仕組みである.
仮に,Head 数が 8 , ${Q}$, ${K}$, ${V}$ の次元数が 512 とすると${\frac{512}{8}=64}$であり,次元数が 64 の ${Q}$, ${K}$, ${V}$ を用いた Scaled Dot-Product Attention を並列に 8 個処理することになる.最終的には個別に計算された値を 1 つのベクトルに落とし込む(concat)ことで単語の分散表現を得る.
この Multi-Head Attention を 1 つのユニット(図 3.2 の Trm )として全結合的に接続したものが BERT モデルである.

\section{Transformer \label{c4s4}}
Transformer とは,再帰型ニューラルネットワーク (以下,RNN) を一切使わずに Attention のみを使うことで,入力と出力の文章同士の広範囲な依存関係を捉えられるモデルである.[7]
RNN は,単語が連続し順序が重要となるような時系列情報を扱うのに最適であるが,逐次的に処理を行うため,処理に多くの時間を必要とする.
また,離れた位置にある文,単語の依存関係をとらえることが難しいといった問題がある.
Transformerは以上のような問題点を克服したモデルである.
Transformerの構造を図aaaに示す.

\begin{figure}[H]%3.3
	\centering
	\includegraphics[width=130mm]{image/transformer.png}
	\caption{Transformerの構造}
	\label{transformer}
\end{figure}

\section{BERT \label{c4s5}}
\subsection{概要}
BERT とは,Bidirectional Encoder Representations from Transformers の略で, 「Transformerによる双方向のエンコード表現」と訳され,2018 年に Jacob Devlin らの論文[6] で発表された自然言語処理モデルである.
また,質問応答 (Question Answering) や自然言語推論 (Multi Natural Language Inference) などの 11 種の自然言語処理タスクにおいて当時の最高性能を達成している手法であり,これ以降,このモデルから派生して作られたモデルが多数存在する.また,現在のWeb検索エンジンにも用いられている.

BERT の学習は,大きく2段階に分けられる.
1つ目がラベル付けされていないデータを学習させる「事前学習」であり,2つ目が事前学習時と比較的少量のデータを用いる「Fine Tuning」である.
事前学習で汎用的なモデルを作成し,Fine Tuningを行うことで,個々のタスクに適応したモデルを作成する.

\subsection{アーキテクチャ}
BERTのモデルアーキテクチャは,双方向のtransformerのエンコーダとなっている.
図aaaにBERTと,BERTと同様に事前学習を利用する自然言語処理モデルを比較したものを示す.

% %\begin{figure}[H]
% 	\centering
% % 	\includegraphics[width=150mm]{image/architechture.png}
% 	\caption{各モデルの構造}
% 	\label{architechture}
% \end{figure}

% \subsubsection{Transformers}
% Transformer とは,再帰型ニューラルネットワーク (以下,RNN) を一切使わずに Attention のみを使うことで,入力と出力の文章同士の広範囲な依存関係を捉えられるモデルである.[7]
% RNN は,単語が連続し順序が重要となるような時系列情報を扱うのに最適であるが,逐次的に処理を行うため,処理に多くの時間を必要とする.
% また,離れた位置にある文,単語の依存関係をとらえることが難しいといった問題がある.
% Transformerは以上のような問題点を克服したモデルである.
% Transformerの構造を図aaaに示す.

% \begin{figure}[H]%3.3
% 	\centering
% 	\includegraphics[width=100mm]{image/transformer.png}
% 	\caption{Transformerの構造}
% 	\label{transformer}
% \end{figure}



\section{LLM \label{c4s6}}
LLMとは、Large Language Model の略称であり、大規模言語モデルと呼ばれる。大規模言語モデルという用語についての正式な定義はないが、大規模コーパスを用いて事前学習を行っており、パラメータ総数が数百万以上の言語モデルを指して言われることが多い。LLMの例として、先述のBERTやOpenAI社が開発したGPT-3が挙げられる。

\section{生成AI \label{c4s7}}
生成AIは、テキストや画像、音声を自律的に生成できるAI技術の総称であり、LLMは生成AIの一部とされる。
生成AIは、文章やテキスト、画像、音声などそれぞれに特化した形で作られている。文章生成であれば先述のGPT-3、GPT-3.5、GPT-4、画像生成であればStable Diffusionといったものが挙げられる。

 % 機械学習
\chapter{機械学習モデリングの課題 \label{c5}}

\section{5.1 \label{c5s1}}
機械学習モデルを構築するためには,学習に必要なデータの確保や,データにラベルを付与するラベリングの工程が必要となる.しかし,グレーゾーンを含む文章だけを取り出すことやラベリングにおいては,人の手を介して行わなければならない.前述の通り,グレーゾーンは18種類存在し,各々のグレーゾーンを含む文章のみを用意することは現実的とは言い難い.

グレーゾーンの一つである「てしまう」を含む文のデータセットの構築について,専門家への聞き取り調査を行った.質問は以下の2点である.
\begin{enumerate}
    \item 作成期間
    \item 作成方法
\end{enumerate}



 % 検証?
\chapter{LLMを用いた話しことば検出}

\begin{comment}
書きことばリストは、話しことばチェッカーで検出しない表現をもとに選んでいる。
それでもchatGPTが出力する場合は、チェッカーの癖もあり得る。← これは要相談
\end{comment} % 
\clearpage
\addcontentsline{toc}{chapter}{参考文献} %章立てせずに目次に追加するおまじない
\renewcommand{\bibname}{参考文献} %これがないと,タイトルが「関連図書」になってしまう
\begin{thebibliography}{99999}
\bibitem{checker}
学生のレポートにおける話し言葉とその出現傾向,日本語日本文学 第 28 号 p57-p71, 
山下由美子,2018



\end{thebibliography}

% \input{chapter/chapter8_thx}

\begin{comment}
label命名規則
1. c[n] : 第n章のラベル
2. c[m]s[n] : 第m章第n節のラベル
\end{comment}


\end{document}